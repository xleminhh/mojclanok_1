% Metódy inžinierskej práce

\documentclass[10pt,twoside,slovak,a4paper]{article}

\usepackage[slovak]{babel}
%\usepackage[T1]{fontenc}
\usepackage[IL2]{fontenc} % lepšia sadzba písmena Ľ než v T1
\usepackage[utf8]{inputenc}
\usepackage{graphicx}
\usepackage{url} % príkaz \url na formátovanie URL
\usepackage{hyperref} % odkazy v texte budú aktívne (pri niektorých triedach dokumentov spôsobuje posun textu)

\usepackage{cite}
%\usepackage{times}

\pagestyle{headings}

\title{Autodesk Maya\thanks{Semestrálny projekt v predmete Metódy inžinierskej práce, ak. rok 2021/22, vedenie: Ing. Fedor Lehocki, PhD.}} % meno a priezvisko vyučujúceho na cvičeniach

\author{Hieu Le Minh\\[2pt]
	{\small Slovenská technická univerzita v Bratislave}\\
	{\small Fakulta informatiky a informačných technológií}\\
	{\small \texttt{xleminhh@stuba.sk}}
	}

\date{\small 25. október 2021} % upravte



\begin{document}

\maketitle

\begin{abstract}
Ja som si vybral na rámcovú tému modelovanie v softvérovom inžinierstve program Autodesk Maya
- je to program využívaný na vytváranie 3d animácií. 
Plánujem sa hlavne zamerať na opis tohto programu, ako funguje, kde sa používa, v akom 
programovacom jazyku sa píše a tak ďalej. Rád by som ešte spomenul výhody a nevýhody, či 
sa tento program oplatí používať a na to by som potom nadviazal porovnanie s inými populárnymi programami
na animácie.
\end{abstract}



\section{Úvod}

V tomto článku sa budeme zaoberať programom Autodesk Maya. Vysvetlím Vám, čo to je za program a potom Vám bližšie tento program opíšem v ďaľších častiach
t.j. napríklad : vytváranie reálneho vlasu, zostavenie svalov, vytvranie hier a programov pomocou unity 3D a Maya atď.

Tu je explicitná štruktúra článku.
Autodesk Maya program je vysvetlený v tejto časti.~\ref{AM}.
použitie programu je v tejto časti.~\ref{pouzitie}.
Záverečné poznámky prináša časť~\ref{zaver}.



\section{Autodesk Maya - čo to je?} \label{AM}

%Z obr.~\ref{f:rozhod} je všetko jasné. 

\begin{figure*}[tbh]
%\centering
%\includegraphics[scale=1.0]{diagram.pdf}
%Aj text môže byť prezentovaný ako obrázok. Stane sa z neho označný plávajúci objekt. Po vytvorení diagramu zrušte znak \texttt{\%} pred príkazom \verb|\includegraphics| označte tento riadok ako komentár (tiež pomocou znaku \texttt{\%}).
Autodesk Maya výkonný softvér, ktorý umožňuje 3D animovanie, modelovanie, simulácie, vykresľovanie a ďalšie. Je robustný a všestranný a mnohí ho považujú za priemyselný štandard pre animácie. Veľa známych filmových štúdií používa Autodesk Maya, vrátane  Blue Sky Studios, Framestore, Moving Picture Company. Tento softvér bol použitý na animáciu známych a ocenených filmov ako sú „Frozen“ a „Wreck It Ralph“.

Pomocou Maya môžete vytvárať 3D prvky pre film, televíziu a videohry. Zvyčajne to zahŕňa niekoľko rôznych komponentov umenia vrátane vytvárania 3D modelov, zostavovania postavy, animácie, dynamiky, maľovania, osvetlenia a vykresľovania. Maya obsahuje intuitívne a ľahko použiteľné nástroje na zjednodušenie všetkých týchto úloh.

Autodesk Maya je dostupný pre operačné systémy Windows, Mac a Linux. Na 1 rok Vás bude tento program stáť 1700 dolárov, čo je v prepočte okolo 1500 eúr.
%\caption{Rozhodujúci argument.}
%\label{f:rozhod}
\end{figure*}



\section{Použitie Autodesk Maya} \label{pouzitie}

Základným problémom je teda\ldots{} Najprv sa pozrieme na nejaké vysvetlenie (časť~\ref{pouzitie:vlas}), a potom na ešte nejaké (časť~\ref{pouzitie:sval})(časť~\ref{pouzitie:unity}).\footnote{Niekedy môžete potrebovať aj poznámku pod čiarou.}

Môže sa zdať, že problém vlastne nejestvuje\cite{Coplien:MPD}, ale bolo dokázané, že to tak nie je~\cite{Czarnecki:Staged, Czarnecki:Progress}. Napriek tomu, aj dnes na webe narazíme na všelijaké pochybné názory\cite{PLP-Framework}. Dôležité veci možno \emph{zdôrazniť kurzívou}.


\subsection{vytváranie reálneho vlasu} \label{pouzitie:vlas}

tu bude clanok



\subsection{vytváranie svalov} \label{pouzitie:sval}

tu bude clanok

\paragraph{Veľmi dôležitá poznámka.}
Niekedy je potrebné nadpisom označiť odsek. Text pokračuje hneď za nadpisom.


\subsection{Unity 3D a Maya} \label{pouzitie:unity}
\begin{enumerate}
\item v hre
Content Creation for a 3D Game with Maya and Unity 3D
\item vo vyucbe
A Gaming Environment to Train Teachers Diagnose Children Learning Disabilities

	%\begin{enumerate}
	%\item x
	%\item y
	%\end{enumerate}
\end{enumerate}


\section{Záver} \label{zaver} % prípadne iný variant názvu



%\acknowledgement{Ak niekomu chcete poďakovať\ldots}


% týmto sa generuje zoznam literatúry z obsahu súboru literatura.bib podľa toho, na čo sa v článku odkazujete
\bibliography{literatura}
\bibliographystyle{plain} % prípadne alpha, abbrv alebo hociktorý iný
\end{document}
