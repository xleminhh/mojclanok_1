\documentclass[10pt,oneside,slovak,a4paper]{article}
\usepackage[slovak]{babel}
\usepackage[IL2]{fontenc}
\usepackage[utf8]{inputenc}
\usepackage{graphicx}
\usepackage{url}
\usepackage{hyperref} 

\usepackage{cite}
%\usepackage{times}

\pagestyle{headings}

\title{Autodesk Maya\thanks{Semestrálny projekt v predmete Metódy inžinierskej práce, ak. rok 2021/22, vedenie: Ing. Fedor Lehocki, PhD.}}

\author{Hieu Le Minh\\[2pt]
	{\small Slovenská technická univerzita v Bratislave}\\
	{\small Fakulta informatiky a informačných technológií}\\
	{\small \texttt{xleminhh@stuba.sk}}
	}

\date{\small 5. november 2021}



\begin{document}

\maketitle

\begin{abstract}
Ja som si vybral na rámcovú tému modelovanie v softvérovom inžinierstve program Autodesk Maya
- je to program využívaný na vytváranie 3d animácií. 
Plánujem sa hlavne zamerať na opis tohto programu, ako funguje, kde sa používa, v akom 
programovacom jazyku sa píše a tak ďalej. Rád by som ešte spomenul výhody a nevýhody, či 
sa tento program oplatí používať a na to by som potom nadviazal porovnanie s inými populárnymi programami
na animácie.
\end{abstract}



\section{Úvod}

Nikto nie je príliš starý na rozprávky a hry. Ešte doteraz si rád pozriem nejakú tú rozprávku, zahrám nejaké tie hry, ale už je to inšie. V poslednej dobe si častejšie všímam a uvažujem nad tým, ako dlho muselo trvať, čo sa muselo použiť na spracovanie takejto rozprávky, filmu alebo hry. Preto som si vybral rámcovú tému Autodesk Maya. Zaujalo ma, ako je možné vytvárať také animácie aké existujú a tu som našiel moju odpoveď.

Tu je explicitná štruktúra článku.
Autodesk Maya softvér je vysvetlený v tejto časti.~\ref{AM}.
Vysvetlenie MEL a Python.~\ref{python}.
Výhody a nevýhody používania tohto programu sa nachádza tu.~\ref{vyhnevyh}.
Porovnanie Maya a Blender.~\ref{porovnanie}.
Použitie Unity 3D a Maya.~\ref{3d}.
Záverečné poznámky prináša časť~\ref{zaver}.



\section{Autodesk Maya - čo to je?} \label{AM}

Autodesk Maya je program, ktorý umožňuje 3D animovanie, modelovanie, simulácie, vykresľovanie a ďalšie. Je všestranný a mnohí ho považujú za priemyselný štandard pre animácie. Veľa známych filmových štúdií používa Autodesk Maya ako sú napríklad Blue Sky Studios, Framestore, Moving Picture Company. Tento softvér bol použitý na animovanie známych a ocenených filmových rozprávok ako sú „Frozen“ a „Wreck It Ralph“.\\

Pomocou Maya môžete vytvárať 3D prvky pre filmy, seriály a dokonca aj pre videohry. Zahŕňa niekoľko rôznych komponentov umenia vrátane vytvárania 3D modelov, zostavovania postavy, animácií, dynamiky, osvetlenia a vykresľovania. Maya obsahuje ľahko použiteľné nástroje na zjednodušenie všetkých týchto úloh.\\

Autodesk Maya je dostupný pre operačné systémy Windows, Mac a Linux. Na 1 rok Vás bude tento program stáť 1700 dolárov, čo je v prepočte okolo 1500 eúr.



\section{MEL a Python} \label{python}
\paragraph{MEL}
celým názvom Maya Embedded Language je skriptový jazyk. Funkcie ako menu, buttony a mnohé iné, ktoré sa v programe nachádzajú, sú napísane cez tento MEL.
\paragraph{Python}
patrí do rebríčka jedných z najpopulárnejších a najobľúbenejších programovacích jazykov. Používa sa napríklad na výrobu rôznych hier, programov a aj na umelú inteligenciu.\\

Prečo som tu spomenul python? MEL je hlavný jazyk programu Autodesk Maya, ale tento program taktiež podporuje písanie v pythone. Výhoda v používaní pythonu je tá, že python narozdiel od MEL má možnosť class managementu, taktiež je štruktúra kódu ľahšie čítateľná a porozumiteľná. MEL má iba funkcie, ktoré súvisia s Mayou narozdiel od pythonu, ktorý je široko podporovaný rôznymi softvérami.\cite{evaluationmaya}



\section{vytváranie reálneho vlasu} \label{pouzitie}

tu bude clanok



\subsection{vytváranie svalov} \label{pouzitie:sval}

tu bude clanok

\paragraph{Veľmi dôležitá poznámka.}
Niekedy je potrebné nadpisom označiť odsek. Text pokračuje hneď za nadpisom.


\subsection{Unity 3D a Maya} \label{pouzitie:unity}
\begin{enumerate}
\item v hre
Content Creation for a 3D Game with Maya and Unity 3D
\item vo vyucbe
A Gaming Environment to Train Teachers Diagnose Children Learning Disabilities

	%\begin{enumerate}
	%\item x
	%\item y
	%\end{enumerate}
\end{enumerate}

\begin{itemize}
\item jedna vec
\item druhá vec
	\begin{itemize}
	\item x
	\item y
	\end{itemize}
\end{itemize}

\section{Záver} \label{zaver} % prípadne iný variant názvu



%\acknowledgement{Ak niekomu chcete poďakovať\ldots}


% týmto sa generuje zoznam literatúry z obsahu súboru literatura.bib podľa toho, na čo sa v článku odkazujete
\bibliography{literatura.bib}
\bibliographystyle{plain} % prípadne alpha, abbrv alebo hociktorý iný
\end{document}
